\newpage
\tableofcontents
\newpage
\section{Множества. Некоторые действия с множествами\\ {\normalfont Объединение, пересечение, дополнение, разность, свойства}}
\subsection{Множества}
\textbf{Множество} — это \sout{множество} набор, совокупность каких-либо (вообще говоря любых) объектов — элементов этого множества.

Множества обычно обозначаются большими латинскими буквами (например, $A$, $B$), его элементы — малыми (то есть $a$, $b$).

\begin{itemize}
    \setlength\itemsep{-0.2em}
    \item Элементы в множестве не повторяются
    \item Два множества равны ($A = B$) тогда и только тогда,\newlineкогда содержат в точности одинаковые элементы.
    \item Порядок элементов в множестве не имеет значения
    \item $\varnothing$ — \textbf{пустое множество}, не содержащее ни одного элемента
\end{itemize}

$a \in A$ — означает, что элемент $a$ принадлежит множеству $A$

$a \notin A$ — что не принадлежит

$A \subset B$ — $A$ это \textbf{подмножество} $B$, то есть $\forall x \in A \Rightarrow x \in B$

$A \subset A$ — всегда верно.

Пусть $a \in A, b \in B$. Упорядоченной парой $(a, b)$ называется множество $\{\{a\}, \{a, b\}\}$, при этом $a$ называется первым элементом упорядоченной пары, а $b$ — вторым.

\subsection{Операции над множествами}

\begin{itemize}
    \setlength\itemsep{-0.2em}
    \item \textbf{Объединение} $A \cup B = \{x: x \in A \vee x \in B\}$
    \item \textbf{Пересечение} $A \cap B = \{x: x \in A \wedge x \in B\}$
    \item \textbf{Разность} $A \setminus B = \{x: x \in A \wedge x \notin B\}$
    \item \textbf{Симметрическая разность} $A \bigtriangleup B = (A \setminus B) \cup (B \setminus A)$
    \item \textbf{Дополнение} $A^C = \{x \in U : x \notin A\} = U \setminus A$
    \item $A \cup \varnothing = A$
    \item $A \cap \varnothing = \varnothing$
    \item $(A \cup B) \cap C = (A \cap C) \cup (B \cap C)$
    \item $(A \cap B) \cup C = (A \cup C) \cap (B \cup C)$
    \item $A \cup B = B \cup A$
    \item $A \cap B = B \cap A$
    \item $A \bigtriangleup B = B \bigtriangleup A$
    \item $(A \cup B)^C = A^C \cap B^C$
    \item $A \cup A^C = U$
    \item $A \cap A^C = \varnothing$
    \item ${(A^C)}^C = A$
    \item $A^C \setminus B^C = B \setminus A$
    \item $\varnothing^C = U$
    \item \textbf{Декартово произведение} $A \times B = \{(a, b) : a \in A \wedge b \in B\}$
\end{itemize}

\section{Действительные числа\\ {\normalfont Ограниченные (сверху, снизу) числовые множества}}

\textbf{Действительное число} (вещественное число) — математический объект, возникший из потребности измерения геометрических и физических величин окружающего мира, а также проведения таких вычислительных операций, как извлечение корня, вычисление логарифмов, решение алгебраических уравнений, исследование поведения функций.

Вещественные числа предназначены для измерения непрерывных величин.

Множество вещественных чисел имеет стандартное обозначение — $\mathbb{R}$.

\subsection{Аксиомы поля}

На множестве $\mathbb{R}$ определено отображение (операция сложения) $+:\mathbb {R} \times \mathbb {R} \to \mathbb {R}$ сопоставляющее каждой упорядоченной паре элементов $a,b$ из $\mathbb{R}$ некоторый элемент $c$ из того же множества $\mathbb{R}$, называемый \textbf{суммой} $a$ и $b$ ($a+b$ эквивалентная запись элемента $c$ множества $\mathbb{R}$).

Также, на множестве $\mathbb{R}$ определено отображение (операция умножения) $\cdot :\mathbb {R} \times \mathbb {R} \to \mathbb {R}$ сопоставляющее каждой упорядоченной паре элементов $a, b$ из $\mathbb{R}$  некоторый элемент $a\cdot b$, называемый \textbf{произведением} $a$ и $b$.

При этом имеют место следующие свойства:
\begin{itemize}
    \setlength\itemsep{-0.2em}
    \item $I_1$. \textit{Коммутативность сложения}. Для любых $a, b \in \mathbb{R}$: $a + b = b + a$
    \item $I_2$. \textit{Ассоциативность сложения}. Для любых $a, b, с \in \mathbb{R}$: $a + (b + c) = (a + b) + c$
    \item $I_3$. \textit{Существование нуля}. Существует элемент $0\in \mathbb {R}$, называемый \textbf{нулём}, такой, что для любого $a \in \mathbb {R}$, $a + 0 = a$
    \item $I_4$. \textit{Существование противоположного элемента}. Для любого  $a \in \mathbb {R}$, существует элемент $-a \in \mathbb{R}$ называемый \textbf{противополжным} к $a$, такой, что $a + (-a) = 0$
    \item $I_5$. \textit{Коммутативность умножения}. Для любых $a, b \in \mathbb{R}$: $a \cdot b = b \cdot a$
    \item $I_6$. \textit{Ассоциативность сложения}. Для любых $a, b, с \in \mathbb{R}$: $a \cdot (b \cdot c) = (a \cdot b) \cdot c$
    \item $I_7$. \textit{Существование единицы}. Существует элемент $1\in \mathbb {R}$, называемый \textbf{единицей}, такой, что для любого $a \in \mathbb {R}$, $a \cdot 1 = a$
    \item $I_8$. \textit{Существование обратного элемента}. Для любого  $a \in \mathbb {R}$, $a \not= 0$, существует элемент $a^{-1} \in \mathbb{R}$ называемый \textbf{обратным} к $a$, такой, что $a \cdot a^{-1} = 1$
    \item $I_9$. \textit{Дистрибутивный закон умножения относительно сложения}. Для любых $a, b, с \in \mathbb{R}$: $a \cdot (b + c) = a \cdot b + a \cdot c$
    \item $I_{10}$ ?. \textit{Нетривиальность поля}. Единица и ноль — различные элементы $\mathbb{R}$: $1 \not= 0$
\end{itemize}

\subsection{Аксиомы порядка}

\begin{itemize}
    \setlength\itemsep{-0.2em}
    \item $II_1$. \textit{Рефлексивность}: $a\leqslant a$
    \item $II_2$. \textit{Антисимметричность}: $(a\leqslant b)\land (b\leqslant a)\Rightarrow (a=b)$
    \item $II_3$. \textit{Транзитивность}: $(a\leqslant b)\land (b\leqslant c)\Rightarrow (a\leqslant c)$
    \item $II_4$. \textit{Линейная упорядоченность}: $(a\leqslant b)\lor (b\leqslant a)$
    \item $II_5$. \textit{Связь сложения и порядка}: $(a\leqslant b)\Rightarrow (a+c\leqslant b+c)$
    \item $II_6$. \textit{Связь умножения и порядка}: $(0\leqslant a)\land (0\leqslant b)\Rightarrow (0\leqslant a\cdot b)$
\end{itemize}

\subsection{Аксиомы непрерывности}

\begin{itemize}
    \item $III_1$. \textit{Аксиома полноты.} Каковы бы ни были непустые множества $A \subset \mathbb{R}$ и $B \subset \mathbb{R}$, такие, что для любых двух элементов $a \in A$ и $b \in B$ выполняется неравенство $a\leqslant b$, существует такое число $\xi \in \mathbb {R}$, что для всех $a\in A$ и $b\in B$ имеет место соотношение $a\leqslant \xi \leqslant b$
\end{itemize}

\subsection{Другие аксиомы}

\begin{itemize}
    \item $IV_1$. \textit{Аксиома Архимеда}: $\forall a, b : a \ll b, \exists n : na > b$
\end{itemize}

Этих аксиом достаточно, чтобы строго вывести все известные свойства вещественных чисел.
\textit{\textbf{Mножеством вещественных чисел называется непрерывное упорядоченное поле.}}

\subsection{Ограниченные числовые множества}

Числовое множество $E \subset \mathbb{R}$ называется \textbf{ограниченным сверху},\newlineесли $\exists b \in \mathbb{R} : \forall x \in E \Rightarrow x \leqslant b$

Числовое множество $E \subset \mathbb{R}$ называется \textbf{ограниченным cнизу},\newlineесли $\exists a \in \mathbb{R} : \forall x \in E \Rightarrow a \leqslant x$
\bigskip

Множество натуральных чисел $\mathbb{N}$ является примером ограниченного снизу множества. Если $a$ принадлежит $\mathbb{R}$ и $b$ принадлежит $\mathbb{R}$, то отрезок $[a, b]$ представляет собой ограниченное множество. Множества рациональных чисел $\mathbb{Q}$, иррациональных чисел $\mathbb{I}$ и $\mathbb{R}$ — примеры неограниченных множеств.

\section{Неограниченность множества натуральных чисел \\ {\normalfont Теорема о числах $2^n$}}

Предположим, что $\mathbb{N}$ ограничено выше. Тогда самая низкая верхняя граница $s$ существует для каждого $n \in \mathbb{N}$. Рассмотрим, что $k$ является наибольшим натуральным числом, которое меньше $s$. Тогда $k + 1 > s$, и $s$ не является верхней границей $\mathbb{N}$, потому что $k + 1$ остаётся натуральным числом. В результате этого противоречия мы можем сделать вывод, что $\mathbb{N}$ не ограничено сверху.
Снизу $\mathbb{N}$ ограниченно $min(\mathbb{N}) = 1$.

$$(a + \alpha)^n > 1 + n\alpha, \alpha \geqslant -1$$

\textbf{Теорема о числах $2^n$}: Среди чисел вида $2^n$ встречаются сколько угодно большие.

\vspace{1.5em}
$(1 + 1)^n > 1 + n$, $2^n > 1 + n$, т.к. $n \in N$ и $\mathbb{N}$  - неограниченное $\Rightarrow 2^n$ — неограниченное
\newpage
\section{Отношение эквивалентности\\ {\normalfont Счётные и несчётные множества}}

\subsection{Эквивалентность}

Бинарное отношение $R$ на множестве $X$ называется отношением эквивалентности, если оно обладает следующими свойствами:
\begin{itemize}
    \setlength\itemsep{-0.2em}
    \item \textit{Рефлексивность}: $\forall x \in X : xRx$.
    \item \textit{Симметричность}: $\forall x,y \in X:  xRy \Rightarrow yRx$.
    \item \textit{Транзитивность}: $\forall x,y,z \in X: xRy \wedge yRz \Rightarrow xRz$.
\end{itemize}

Отношение эквивалентности обозначают символом $\sim$. Запись вида $a \sim b$ читают как "$a$ эквивалентно $b$".

Два множества называются \textbf{эквивалентными} (или \textbf{равномощными}), если между ними можно установить \textit{взаимно однозначное соответствие}.

\textbf{Примеры}: отношение \textit{равенства} в любом множестве, \textit{параллельности} прямых на плоскости, \textit{быть одного роста} на множестве людей.

\subsection{Счётность}

Множества, состоящие из конечного числа элементов, называются \textbf{конечными}, а состоящие из бесконечного числа — \textbf{бесконечными}.

Множество, эквивалентное множеству $\mathbb{N}$ натуральных чисел, называется \textbf{счетным}.

\textbf{Теорема}: Множество $\mathbb{Z}$ целых чисел счетно.

\textbf{Док-во}: Можно поставить в соответствие каждому натуральному числу $n$ число $z_n = \frac{n}{2}$, если $n$ — четное, и $z_n = -\frac{n-1}{2}$, если $n$ — нечетное. Данное соответствие сопоставляет каждому натуральному числу $n$ целое число $z_n$, причем каждое из целых чисел получается по этой формуле ровно один раз.

\begin{itemize}
    \setlength\itemsep{-0.2em}
    \item Любое бесконечное множество содержит счетное подмножество.
    \item Любое бесконечное подмножество счетного множества счетно.
    \item Объединение конечного и счетного множеств, объединение двух счетных множеств — счетные.
    \item Множество $\mathbb{Q}$ рациональных чисел счетно.
    \item Декартово произведение конечного числа не более чем счётных множеств — не более чем счётно.
    \item Множество, не являющееся ни конечным, ни счетным, называется \textbf{несчетным} множеством.
    \item Множество $\mathbb{R}$ действительных чисел несчетно.
    \item Мощность множества действительных чисел также называют \textbf{континуумом}, и по сравнению со счётными множествами это «более бесконечное» множество.
\end{itemize}
\newpage

\section{Модуль действительного числа\\ {\normalfont Свойства}}
\subsection{Модуль}

\textbf{Модулем}, или абсолютной величиной, числа $a \in \mathbb{R}$ называется число $\lvert a\rvert \in \mathbb{R}$, равное самому $a$, если $a$ неотрицательно, и равное $-a$, если $a$ отрицательно:

\[ \lvert a\rvert =
  \begin{cases}
    a       & \quad a \geqslant 0,\\
    -a      & \quad a < 0.
  \end{cases}
\]

Из определения модуля $a$ ясно, что $\lvert a\rvert$ — неотрицательное число.

\subsection{Свойства}

\begin{itemize}
    \setlength\itemsep{-0.2em}
    \item $\lvert x\rvert = \lvert-x\rvert$
    \item $\lvert x\rvert \leqslant a \Leftrightarrow -a \leqslant x \leqslant a$
    \item $\lvert x+y\rvert \leqslant \lvert x\rvert + \lvert y\rvert$
    \item $\lvert \lvert x\rvert - \lvert y\rvert\rvert \leqslant \lvert x - y\rvert$
    \item $\lvert x - y\rvert = \lvert y - x\rvert$
\end{itemize}

\section{Комплексные числа\\ {\normalfont Определение, арифм. действия, алгебраическая форма}}

\textbf{Комплексные числа} — это расширение поля действительных чисел. Обозначается $\mathbb {C}$. Комплексные числа образуют алгебраически замкнутое поле, то есть многочлен степени $n$ с комплексными коэффициентами имеет ровно $n$ комплексных корней. Это \textbf{основная теорема алгебры}.

Формально, комплексное число $z$ — это упорядоченная пара вещественных чисел $(x, y)$ с введёнными на них операциями сложения и умножения вида:

$$(x_1, y_1) + (x_2, y_2) = (x_1 + x_2, y_1 + y_2)$$
$$(x_1, y_1) \cdot (x_2, y_2) = (x_1x_2 - y_1y_2, x_1y_2 + x_2y_1)$$

Арифметическая модель комплексных чисел как пар действительных чисел, предложенная У. Р. Гамильтоном, хотя и непротиворечива, но не удобна в вычислениях, поэтому для манипуляций с ними используют различные их представления.

В рамках гамильтоновского определения действительные числа имеют вид $(x, 0)$. Эта пара обозначается также просто $x$. В частности, $(0, 0) = 0$. Пара $(0, 1)$ имеет особый статус и называется \textbf{мнимой единицей}. Она обозначается как $i$.

$$i^2 = i\cdot i = (0, 1) \cdot (0, 1) = (0 - 1, 0 + 0) = -1$$

\textbf{Алгебраическая форма} комплексного числа: $(x, y) = x + i\cdot y$.

\textbf{Сложение} двух комплексных чисел в алгебраической $z_1 = x_1 + y_1i$ и $z_2 = x_2 + y_2i$ выполняется по следующему правилу: $z_1 + z_2 = (x_1 + x_2) + (y_1 + y_2)i$, \textbf{произведение} аналогично $z_1 \cdot z_2 = (x_1x_2 - y1_y2) + (x_1y_2 + x_2y_1)i$

\section{Модуль и аргумент комплексного числа\\ {\normalfont Тригонометрическая форма записи}}
\subsection{Тригонометрическая форма комплексного числа}

Каждому комплексному числу $z = x + iy$ геометрически соответствует точка $M(x,y)$ на плоскости $Oxy$. Но положение точки на плоскости, кроме декартовых координат $(x,y)$, можно зафиксировать другой парой — ее полярных координат $(r,\varphi)$ в полярной системе.

Используя связь декартовых и полярных координат точки $M\colon \begin{cases} x=r\cos\varphi,\\ y=r\sin\varphi\end{cases}$, из алгебраической формы записи комплексного числа $z=x+iy$ получаем \textbf{тригонометрическую форму}:

$$z=r \bigl(\cos\varphi+i\sin\varphi\bigr)$$

\textbf{Умножение} комплексных чисел в тригонометрической форме:
$$z_1\cdot z_2= r_1\cdot r_2\cdot \bigl(\cos(\varphi_1+\varphi_2)+ i\sin(\varphi_1+ \varphi_2)\bigr).$$

\textbf{Возведение в степень}: $z^n= r^n(\cos n\varphi+ i\sin n\varphi)$

\subsection{Показательная форма комплексного числа}

Если обозначить комплексное число $z$, у которого $\operatorname{Re}z= \cos\varphi$, а $\operatorname{Im}z=\sin\varphi$, через $e^{i\,\varphi}$, то есть $\cos\varphi+i\sin\varphi=e^{i\,\varphi}$, то получим \textbf{показательную форму} записи комплексного числа:

$$z=r\cdot e^{i\,\varphi}$$

Равенство $e^{i\,\varphi}= \cos\varphi+i\sin\varphi$ называется \textbf{формулой Эйлера}.

Заметим, что геометрически задание комплексного числа $z=(r,\varphi)$ равносильно заданию вектора $\overrightarrow{OM}$, длина которого равна $r$, то есть $\bigl|\overrightarrow{OM}\bigr|=r$, а направление — под углом $\varphi$ к оси $Ox$.

\subsection{Модуль и аргумент комплексного числа}

Число $r$ — длина радиуса-вектора точки $M(x,y)$ называется \textbf{модулем} комплексного числа $z=x+iy$. Обозначение: $|z| = r$.

$$|z|=\sqrt{x^2+y^2}$$

Геометрический смысл модуля комплексного числа
Очевидно, что $|z|\geqslant0$ и $|z|=0$ только для числа $z=0~(x=0,\,y=0)$.
Число $|z_1-z_2|$ есть расстояние между точками $z_1$ и $z_2$ на комплексной плоскости.

Полярный угол $\varphi$ точки $M(x,y)$ называется \textbf{аргументом} комплексного числа $z=x+iy$. Обозначение: $\varphi=\arg z$.

В дальнейшем, если нет специальных оговорок, под $\arg z$ будем понимать значение $\varphi$, удовлетворяющее условию $-\pi<\varphi\leqslant\pi$. Так, для точки $z=-1-i$ $\arg z=-\frac{3\pi}{4}$.

Аргумент числа $z = 0$ — величина неопределенная.

Для пары сопряженных комплексных чисел $z$ и $\overline{z}$ справедливы следующие равенства: $|\overline{z}|= |z|,\qquad \arg\overline{z}=-\arg z$

\section{Формула Муавра\\ {\normalfont Извлечение корней из комплексного числа}}

\textbf{Формула Муавра} для комплексных чисел $z=r(\cos \varphi +i\sin \varphi)$ утверждает, что

$$z^{n}=r^{n}(\cos \varphi +i\sin \varphi )^{n}=r^{n}(\cos n\varphi +i\sin n\varphi)$$

для любого $n\in \mathbb{N}$.

Исторически формула Муавра была доказана ранее \textbf{формулы Эйлера}:
$e^{ix}=\cos x+i\sin x,$ однако немедленно следует из неё.

Аналогичная формула применима также и при вычислении корней n-й степени из ненулевого комплексного числа:

$$z^{1/n}={\big [}r{\big (}\cos(\varphi +2\pi k)+i\sin(\varphi +2\pi k){\big )}{\big ]}^{1/n}=r^{1/n}\left(\cos {\frac {\varphi +2\pi k}{n}}+i\sin {\frac {\varphi +2\pi k}{n}}\right),$$
где $k=0,1,\dots ,n-1$

Из этой формулы следует, что корни $n$-й степени из ненулевого комплексного числа всегда существуют, и их количество равно $n$. На комплексной плоскости, как видно из той же формулы, все эти корни являются вершинами правильного n-угольника, вписанного в окружность радиуса $\sqrt[ {n}]{r}$ с центром в нуле.

\section{Линейное, Евклидовое пространство\\ {\normalfont Неравенство Коши-Буняковского}}
\subsection{Линейное пространство}

\textbf{Линейным} (\textbf{векторным}) пространством называется множество $V$ произвольных элементов, называемых \textbf{векторами}, в котором определены операции \textit{сложения векторов} и \textit{умножения вектора на число}, т.е. любым двум векторам $\mathbf{u}$ и ${\mathbf{v}}$ поставлен в соответствие вектор $\mathbf{u}+\mathbf{v}$, называемый \textit{суммой векторов} $\mathbf{u}$ и ${\mathbf{v}}$, любому вектору ${\mathbf{v}}$ и любому числу $\lambda$ из поля действительных чисел $\mathbb{R}$ поставлен в соответствие вектор $\lambda\cdot\mathbf{v}$, называемый произведением вектора $\mathbf{v}$ на число $\lambda$; так что выполняются 8 \textbf{аксиом линейного пространства}:\textit{ коммутативность и ассоциативность сложения, существование нулевого вектора, существование противоположного вектора, унитарность, ассоциативность умножения, дистрибутивность относительно сложения векторов и скаляров}.

\begin{itemize}
    \setlength\itemsep{-0.2em}
    \item Линейное пространство — это непустое множество, так как обязательно содержит единственный нулевой вектор.
    \item Векторное пространство является абелевой группой по сложению.
\end{itemize}

\textbf{Скалярное произведение}: $(\mathbf {a} ,\mathbf {b} )=|\mathbf {a} ||\mathbf {b} |\cos(\theta).$


\subsection{Евклидово пространство}

Вещественное линейное пространство $\mathbb{E}$ называется евклидовым, если каждой паре элементов $\mathbf{u},\,\mathbf{v}$ этого пространства поставлено в соответствие действительное число $\langle\mathbf{u},\mathbf{v} \rangle$, называемое скалярным произведением, причем это соответствие удовлетворяет следующим условиям:

$$\begin{aligned} &\bold{1.}\quad \langle\mathbf{u},\mathbf{v}\rangle= \langle\mathbf{v}, \mathbf{u}\rangle\quad \forall \mathbf{u},\mathbf{v}\in \mathbb{E}\,;\\[2pt] &\bold{2.}\quad \langle\mathbf{u}+\mathbf{v},\mathbf{w}\rangle= \langle\mathbf{u}, \mathbf{w}\rangle+ \langle\mathbf{v},\mathbf{w}\rangle\quad \forall \mathbf{u},\mathbf{v},\mathbf{w}\in \mathbb{E}\,;\\[2pt] &\bold{3.}\quad \langle \lambda\cdot \mathbf{u},\mathbf{v}\rangle= \lambda\cdot \langle\mathbf{u},\mathbf{v}\rangle\quad \forall \mathbf{u},\mathbf{v}\in \mathbb{E},~~ \forall \lambda\in \mathbb{R}\,;\\[2pt] &\bold{4.}\quad \langle\mathbf{v},\mathbf{v}\rangle>0\quad \forall \mathbf{v}\ne \mathbf{o}~\land~ \langle\mathbf{v},\mathbf{v}\rangle=0~~ \Rightarrow~~ \mathbf{v}=\mathbf{o}\,.\end{aligned}$$

\subsection{Неравенство Коши — Буняковского}

Пусть дано линейное пространство $L$ со скалярным произведением $\langle x,\;y\rangle$. Пусть $\|x\|$ — норма, порождённая скалярным произведением, то есть $\|x\|\equiv\sqrt{\langle x,\;x\rangle},\;\forall x\in L$. Тогда для любых $x,\;y\in L$ имеем:

$$|\langle x,\;y\rangle| \leqslant \|x\|\cdot\|y\|$$,
причём равенство достигается тогда и только тогда, когда векторы $x$ и $y$ линейно зависимы (коллинеарны, или среди них есть нулевой).

\section{Метрическое пространство, открытый замкнутый шар\\ {\normalfont Открытое, замкнутое множество}}

\subsection{Метрика и метрическое пространство}

Пусть $X$ — абстрактное множество.

$X \times X=\{(x_1,x_2): x_i \in X\}$ — прямое произведение множества $X$ на себя

Отображение $\rho : X \times X \rightarrowmathbb{R}^{+}$ — называется \textbf{метрикой} на $X$, если выполняются аксиомы
\begin{enumerate}
    \setlength\itemsep{-0.2em}
    \item $\rho(x,y) \geqslant 0; \rho(x,y)=0 \Leftrightarrow x=y$
    \item $\rho(x,y) = \rho(y,x)$
    \item $\rho(x,y)\leqslant\rho(x,z)+\rho(z,y)$ — неравенство треугольника
\end{enumerate}

Если на $X$ определена метрика, то пара $(X, \rho)$ называется \textbf{метрическим пространством}.

\subsection{Открытые шары}

Для метрических пространств основное значение имеют открытые шары.

Пусть $(X,\rho)$ — метрическое пространство, пусть $r \in \mathbb{R}, r > 0, a \in X$, тогда открытый шар радиуса $r$ в точке $a$ — это множество $V_r(a)=\{x \in X : \rho(x,a) < r\}$

\textbf{Пример открытого шара}. На числовой оси: $X = \mathbb{R} : V_r(a)=(a−r; a+r)$

Множество $M \subset X$ \textbf{ограничено}, если существуют $a \in X$ и $r \in (0; +\inf)$, такие, что $M \subset V_r(a)$. Иначе говоря, множество ограничено, если его \textit{можно поместить в открытый шар конечного радиуса}.

\subsection{Открытые множества}

Множество $G \subset X$ называется открытым в метрическом пространстве, если его можно записать как некоторое объединение открытых шаров (в общем случае объединение может состоять из несчетного числа шаров).

    $\tau$ — класс открытых множеств.
    
    $\tau ={G — \text{открытые в МП} (X,\rho)}$

\subsection{Замкнутые множества}

Множество F называется замкнутым в МП$(X,\rho)$, если $\overline{F} = X \setminus F$ — открыто.

\textbf{Свойства замкнутых множеств:}
\begin{enumerate}
    \setlength\itemsep{-0.2em}
    \item $X, \varnothing$ — замкнуты
    \item Если $F_{\alpha}$ — замкнуто $\forall \alpha \in A$, то $\cap_{\alpha \in A}F_{\alpha}$ — замкнуто
    \item Если $F_1…F_n$ — замкнуты, то $\cup_{j=1}^{n}F_j$ — замкнуто
\end{enumerate}

\section{Нормированное пространство}

-

\section{Определение предела числовой последовательности}

-

\section{Единственность предела числовой \mbox{последовательности}}

-

\section{Критерий Коши о существовании предела числовой последовательности}

-

\section{Теорема об ограниченности сходящихся \newlineпоследовательностей}

-

\section{Теорема о предельном переходе неравенства}

-

\section{Теорема о двух полицейских}

-

\section{Теорема Вайерштрасса (без доказательства)\\ {\normalfont Число $e$}}

-

\section{Бесконечно малые последовательности}

-

\section{Числовой ряд и его сумма \\{\normalfont Критерий Коши его существования}}

-
